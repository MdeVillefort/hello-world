% I wrote this in TeXworks.

\documentclass{article}

\usepackage[margin = 2cm]{geometry} % sets margins of document to 2 cm
\usepackage[colorlinks = true, linkcolor = red]{hyperref} % creates internal links whenever \ref is used
\usepackage{mdframed} % includes the mdframed env for boxes of text
\usepackage{verbatim} % includes comment env for multi line comments
\usepackage{graphicx} % includes figure env
\usepackage{amsthm} % includes \theoremstyle command to modify sytle of numbered env, particularly \newtheorem

\setlength{\parindent}{0cm} % set paragraph indent to zero throughout document
\newcommand{\ind}{\setlength{1cm}} % creates custom \ind length for paragraph indents

\theoremstyle{definition} % the following 4 lines set up my definition and proposition numbered envs
\newtheorem{definition}{Definition}[section]
\theoremstyle{definition}
\newtheorem{proposition}{Proposition}[section]

\title{
\Huge Definitions and Theorems\\
\Large Advanced Linear Systems Theory
}
\author{Ethan Ross}
\date{}

\begin{document}
\maketitle
\vspace{2cm}

\begin{figure}[h]
	\centering
	\includegraphics[scale = 0.4]{UOA_Logo.png}
\end{figure}

\clearpage
\tableofcontents
\clearpage

\section{Vector Spaces} \label{ch:1}

\subsection{$R^n$ and $C^n$} \label{sec:1A}

\section{Eigenvalues, Eigenvectors, and Invariant Subspaces} \label{ch:5}

\subsection{Invariant Subspaces} \label{sec:5A}

\begin{definition} \label{def:5.2}
	\textit{ invariant subpaces}\\
	Suppose $T \in \mathcal{L}(V)$.  A subspace $U$ of $V$ is called \emph{invariant} under $T$ if $u \in U$ implies $Tu \in U$.
\end{definition}

\begin{definition} \label{def:5.5}
	\textit{eigenvalue}\\
	Suppose $T \in \mathcal{L}(V)$.  A number $\lambda \in F$ is called an \emph{eigenvalue} of $T$ if there exists $v \in V$ such that $v \neq 0$ and $Tv = \lambda v$.

	Note: eigenvalues are intimately connected with 1D invariant subspaces.
\end{definition}

\begin{proposition} \label{prop:5.6}
	\textit{Equivalent conditions to be an eigenvalue}\\
	Suppose $V$ is finite-dimensional, $T \in \mathcal{L}(V)$, and $\lambda \in F$.  Then the following are equivalent:

	\begin{enumerate}
		\item $\lambda$ is an eigenvalue of $T$;

		\item $T - \lambda I$ is not injective;

		\item $T - \lambda I$ is not surjective;

		\item $T - \lambda I$ is not invertible.
	\end{enumerate}
\end{proposition}

\begin{definition} \label{def:5.7}
	\textit{eigenvector}\\
	Suppose $T \in \mathcal{L}(V)$ and $\lambda \in F$ is an eigenvalue of $T$.  A vector $v \in V$ is called an \emph{eigvenvector} of $T$ corresponding to $\lambda$ if $v \neq 0$ and $Tv 		= \lambda v$.

	Put another way, the vector $v \in V$ with $v \neq 0$ is an eigenvector of $T$ corresponding to $\lambda$ if and only if $v \in null(T - \lambda I)$.
\end{definition}

\begin{proposition} \label{prop:5.10}
	\textit{Linearly independent eigenvectors}\\
	Let $T \in\ \mathcal{L}(V)$.  Suppose $\lambda_1,..., \lambda_m$ are distinct eigenvalues of $T$ and $v_1,..., v_m$ are corresponding eigenvectors.  Then $v_1,..., v_m$ is linearly 			independent.
\end{proposition}

\begin{proposition} \label{prop:5.13}
	\textit{Number of eigenvalues}\\
	Suppose $V$ is finite-dimensional.  Then each operator on $V$ has at most dim $V$ distinct eigenvalues.
	\medskip

	\textbf{Proof}:
\end{proposition}

\end{document}

